\section{Построяване на диференчна схема}
Искаме да построим диференчна схема с втори ред на точност, т.е. $O(h^2 + \tau)$, където $h$ - стъпката на дискретизацията по пространството и $\tau$  - стъпка на дискретизация по времето.
Това създава съответни особености, предвид прекъснатото решение в точката на контакт.

\noindent Разглеждаме (точното) решението като една функция $u(x)$:
\begin{equation}
    u(x)=
        \begin{cases}
            u_1(x) & \text{ако } x \in (-\infty, 0)\\
            u_2(x) & \text{ако } x \in [0, +\infty)
        \end{cases}
\end{equation}
Приближеното решение за $u(x)$ в точката $(x_i, t_j)$ от мрежата ще бележим с $y_{i}^{j}$. ,,Безкрайността`` ще апроксимираме с т.нар. ,,актуална безкрайност``, т.е. ще разглеждаме достатъчно голям интервал $[-M, M]$, за някое положително число $M$.

\noindent Въвеждаме мрежата:
\begin{equation*}
    \omega_{h, \tau } = \left\{ (x_i, t_j):  x_i = -M + (i-1) h, t_j = (j-1) \tau; i = \overline{1,2n+1}, j = \overline{1,m+1};  n = \left\lceil \frac{2M}{h} \right\rceil, m =\left\lceil \frac{T}{\tau} \right\rceil \right\}
\end{equation*}

\noindent Условията за устойчивост на схемата (т.е за $h$ и $\tau$) ще определим след малко.
Да забележим, че при така построената мрежа, на $j$-тия слой по времето, стойността на приближенето решение в точката на котакт ще бъде $y_{n+1}^j$.

Можем вече да апроксираме основните диференциални уравнения. Използваме формулата с разлика напред за производните по времето и формулата с втори ред на точност за втората производна.
С изключение на точките на контакт това е стандартна диференчна схема с грешка $O(h^2+\tau)$ за линейна дифузионна задача. Затова можем директно да запишем:
\begin{align}
    y_{i}^{j+1} &= \left(1-\frac{2 \tau \kappa_1}{h^2}\right)y_{i}^j + \frac{\tau \kappa_1}{h^2}\left(y_{i-1}^j + y_{i+1}^j\right); j = \overline{1, m}; i  = \overline{2, n} \\
    y_{i}^{j+1} &= \left(1-\frac{2 \tau \kappa_2}{h^2}\right)y_{i}^j + \frac{\tau \kappa_2}{h^2}\left(y_{i-1}^j + y_{i+1}^j\right); j = \overline{1, m}; i  = \overline{n+2, 2n}
\end{align}
На база тези две диференчни уравнения ще получим и условията за устойчивост на схемата. Избираме:
\begin{align*}
    \tau &< \frac{h^2}{2 d}\\
    d &:= max \left\{\kappa_1, \kappa_2 \right\}
\end{align*}
Остава да апроксираме с втори ред на точност задачата и в особената точка.

За целта ще се върнем към оригиналната дефиниция на задачата и като начало ще апроксираме производните в условието за идеален контакт с формула с  грешка $O(h+\tau)$ като начало:
\begin{align*}
    \frac{\partial u_1}{\partial x} (0, t) &= \frac{k_2}{k_1} \frac{\partial u_2}{\partial x} (0, t) \\
    \phi_{n+1}^{j} &= \psi_{n+1}^{j} \\
    \frac{\phi_{n+1}^{j} - \phi_{n}^{j}}{h} &=  \frac{k_2}{k_1} \frac{\psi_{n+2}^{j} - \psi_{n+1}^{j}}{h}
\end{align*}
Където $\phi_i^j$ и $\psi_i^j$ са съотвените приближени решения за $u_1(x,t)$ и $u_2(x,t)$.
За безизточниковото уравнение на дифузията при дифузионен коефициент \textbf{1}, лесно се показва (чрез развиване в ред на Тейлър) и допускане, че основното диференциално уравнение
е изпълнено с достатъчно добра точност и върху границите (съображения за гладкост на решението), че трябва от лявата страна на горното диференчно уравнение да извадим $\frac{\partial u_1}{\partial t}$, a от дясната: $-\frac{k_2}{k_1}\frac{\partial u_2}{\partial t}$.
Дискертизираме производните по времето в тези допълнителни членове с формулата с разлика напред и получаваме:
\begin{align*}
    \phi_{n+1}^{j} &= \psi_{n+1}^{j} \\
    \left(\frac{\phi_{n+1}^{j} - \phi_{n}^{j}}{h} - \frac{\phi_{n+1}^{j+1}-\phi_{n+1}^{j}}{\tau}\right) &=  \frac{k_2}{k_1} \left(\frac{\psi_{n+2}^{j} - \psi_{n+1}^{j}}{h} + \frac{ \psi_{n+1}^{j+1}-\psi_{n+1}^{j}}{\tau}\right)
\end{align*}
С помощта на системата \textit{Mathematica} извършваме необходимите алгебрични преобразувания и се връщаме към нотацията на ,,общо`` приближено решение $y_i^j$:
\begin{equation}
    y_{n+1}^{j+1} = \frac{2 \tau}{h^2} \frac{k_1 \kappa_1 \kappa_2 }{\kappa_2 k_1 + \kappa_1 k_2} y_{n+1}^j + \left( 1 - \frac{2 \tau}{h^2}  \frac{(k_1 + k_2) \kappa_1 \kappa_2 }{\kappa_2 k_1 + \kappa_1 k_2} \right) y_{n+1}^j + \frac{2 \tau}{h^2} \frac{k_2 \kappa_1 \kappa_2 }{\kappa_2 k_1 + \kappa_1 k_2} y_{n+2}^j
\end{equation}
Горното е за $j > 1$, т.е. за слоеве по времето след първия (определен от началното условие).
Тогава можем да обобщим всички разсъждения дотук в следната диференчна схема, с която ще получаваме приближеното решение на задачата:
\begin{align*}
    y_i^1 &= 0, i = \overline{1,n} \\
    y_i^{1} &= u_0, i = \overline{n+1, 2n+1} \\
    \\
    За j &= \overline{2,m+1}: \\
    y_{i}^{j+1} &= \left(1-\frac{2 \tau \kappa_1}{h^2}\right)y_{i}^j + \frac{\tau \kappa_1}{h^2}\left(y_{i-1}^j + y_{i+1}^j\right);  i  = \overline{2, n} \\
    y_{n+1}^{j+1} &= \frac{2 \tau}{h^2} \frac{k_1 \kappa_1 \kappa_2 }{\kappa_2 k_1 + \kappa_1 k_2} y_{n+1}^j + \left( 1 - \frac{2 \tau}{h^2}  \frac{(k_1 + k_2) \kappa_1 \kappa_2 }{\kappa_2 k_1 + \kappa_1 k_2} \right) y_{n+1}^j + \frac{2 \tau}{h^2} \frac{k_2 \kappa_1 \kappa_2 }{\kappa_2 k_1 + \kappa_1 k_2} y_{n+2}^j \\
    y_{i}^{j+1} &= \left(1-\frac{2 \tau \kappa_2}{h^2}\right)y_{i}^j + \frac{\tau \kappa_2}{h^2}\left(y_{i-1}^j + y_{i+1}^j\right);  i  = \overline{n+2, 2n} \\
    y_{1}^{j+1} &= 0  \\
    y_{2n+1}^{j+1} &= u_0 \\
\end{align*}

Така написаната диференчна схема може директно да бъде инплементирана например в \textit{Mathematica}. Кодът използван за следващите числени експерименти може да бъден намерен във файла \textit{mmieproj-code.nb}.