\section{Построяване на диференчна схема}
Искаме да построим диференчна схема с втори ред на точност, т.е. $O(h^2 + \tau)$, където $h$ - стъпката на дискретизацията по пространството и $\tau$  - стъпка на дискретизация по времето.
Това създава съответни особености, предвид прекъснатото решение в точката на контакт.

\noindent Разглеждаме (точното) решението като една функция $u(x)$:
\begin{equation}
    u(x)=
        \begin{cases}
            u_1(x) & \text{ако } x \in (-\infty, 0)\\
            u_2(x) & \text{ако } x \in [0, \infty)
        \end{cases}
\end{equation}
Приближеното решение за $u(x)$ в точката $(x_i, t_j)$ от мрежата ще бележим с $y_{i}^{j}$. ,,Безкрайността`` ще апроксимираме с т.нар. ,,актуална безкрайност``, т.е. ще разглеждаме достатъчно голям интервал $[-M, M]$, за някое положително число $M$.

\noindent Въвеждаме мрежата:
\begin{equation*}
    \omega_{h, \tau } = \left\{ (x_i, t_j):  x_i = -M + i h, t_j = j \tau; i = \overline{0,2n}, j = \overline{0,m};  n = \left\lceil \frac{2M}{h} \right\rceil, m =\left\lceil \frac{T}{\tau} \right\rceil \right\}
\end{equation*}

\noindent Условията за устойчивост на схемата (т.е за $h$ и $\tau$) ще определим след малко.
Да забележим, че при така построената мрежа, на $j$-тия слой по времето, стойността на приближенето решение в точката на котакт ще бъде $y_{n+1}^j$.

Можем вече да апроксираме основните диференциални уравнения. Използваме формулата с разлика напред за производните по времето и формулата с втори ред на точност за втората производна.
С изключение на точките на контакт това е стандартна диференчна схема с грешка $O(h^2+\tau)$ за линейна дифузионна задача. Затова можем директно да запишем:
\begin{align}
    y_{i}^{j+1} &= \left(1-\frac{2 \tau \kappa_1}{h^2}\right)y_{i}^j + \frac{\tau \kappa_1}{h^2}\left(y_{i-1}^j + y_{i+1}^j\right); j = \overline{1, m}; i  = \overline{2, n} \\
    y_{i}^{j+1} &= \left(1-\frac{2 \tau \kappa_2}{h^2}\right)y_{i}^j + \frac{\tau \kappa_2}{h^2}\left(y_{i-1}^j + y_{i+1}^j\right); j = \overline{1, m}; i  = \overline{n+2, 2n}
\end{align}
На база тези две диференчни уравнения ще получим и условията за устойчивост на схемата. Избираме:
\begin{align*}
    \tau &< \frac{h^2}{2 d}\\
    d &:= max \left\{\kappa_1, \kappa_2 \right\}
\end{align*}
Остава да апроксираме с втори ред на точност задачата и в особената точка.