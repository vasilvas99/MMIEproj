\section{Числени експерименти}
Тъй като най-добре резултатите от изчисленията за такъв динамичен процес се визуализират като анимация, то тук ще бъдат дадени референции към части от notebook-файла, в който е имплементирано решението.
Съответните секции ще бъдат отбелязани с коментар в \textit{Mathematica} - \textcolor{comment}{(*Коментар*)}.
Точното решение има особеност за $t = 0$, всички анимации визуализират $t>0$.
\subsection{Две тела с еднакви топлофизични характеристики}
\subsubsection{Мед-мед}
Ще разгледаме първо контакта на две тела от мед - добър топлопроводник (вж. \autoref{fig:constants}). Анимациите, визуализиращи решението и грешката се намират в секцията  \textcolor{comment}{(*Copper - Copper contact*)}.

Добър избор за време за изчисленията и актуална безкрайност са $T_{max} = 20~s$, $X_{max} = 50~cm$. Резултатът от пресмятанията е очакваният такъв - получаваме решението за дифузионната задача за едно тяло, с н.у. функцията на
Хевисайд (функция-стъпка).

В началото грешката е голяма, от една страна поради особеността на точното решение, а от друга - заради факта, че дифузията все още не е ,,загладила`` профила на решението. След малък интервал от време, двете решения - точно и числено стават напрактика неразличими. Във визуализацията на грешката
се вижда, как тя бързо и рязко намалява, като ,,дифундира`` в областта. Това е очаквано, тъй като дифузионният оператор е линеен.

Също очаквано - топлината в добър топлопроводник се разпространява бързо, което е и част от обяснението за описаното бързо намаляване на разликата между точното и приближено решение.

\subsubsection{Дърво-Дърво}
\label{sect:WoodWood}
Тук секцията е \textcolor{comment}{(*Wood - Wood contact*)}.
 Разсъжденията и в този случай са аналогични на горния, като единствено актуалната безкрайност е намалена до $X_{max}=20~cm$. 
 
Очаквано за изолатор, топлината се разпространява значително по-бавно - съответно профила на решението се заглажда доста по-бавно, както и грешката
намалява по-бавно.

\subsection{Две тела с различни топлофизични характеристики}

\subsubsection{Мед-дърво и дърво-мед}
Съответните секции са \textcolor{comment}{(*Copper - Wood contact*)} и \textcolor{comment}{(*Wood - Copper contact*)}. 
Тоест кое е ,,студеното`` и кое - ,,топлото`` тяло има значение за решението на задачата.
За времето за пресмятане и актуална безкрайност избираме същите като в \autoref{sect:WoodWood}. 

В предните два случая, може да се види лесно, че точката на контакт е с температурата приблизително $\frac{u_0}{2}$. В двата случая на тела с разнородни топлофизични характеристики,
температурата на точката на контакт е по-близо до температурата на медта (по-добрият топлопроводник). 

Другият интересен факт е, че температурата на медта на практика е постоянна и равна на началната в целия интервал, в който разглеждаме задачата, а в дървото виждаме ясен дифузионен режим на разпространение на топлината.
Това е физически смислен резултат - топлината в по-добрия топлопроводник се разпространява толкова бързо, че за времето, в което топлината започне да се разпространява
в дървото, всякакви температурни разлики в медта са компенсирани.
Това се вижда добре и от грешката - през цялото време в медта грешката е на практика 0, тъй като в нея температурата остава почти непроменена, а в дървото и грешката ,,дифундира`` бавно в интервала.

\subsubsection{Гранит-дърво и дърво-гранит}
Ще изберем два изолатора, но двата имат порядък разлика в температуропроводността. Секциите са \textcolor{comment}{(*Granite - Wood contact*)} и \textcolor{comment}{(*Wood - Granite contact*)}.

Ситуацията е междинна на горните - за два еднакви и два със значително различни топлофизични характеристики материала. Точката на контакт отново е с температура по-близка до тази на материала с по-голяма температуропроводност - в случая гранита.
Тук и в двата материала има видим дифузионен профил на разпространението на топлината, намаляването на грешката също е относително симетрично. 

\subsubsection{Мед-чугун и чугун-мед}
Секциите са: \textcolor{comment}{(*Copper - Cast Iron*)} и \textcolor{comment}{(*Cast Iron - Copper contact*)}. 
Тук пък имаме добри топлопроводници, които имат температуропроводност с порядък разлика. Резултатът до голяма степен преповтаря горните. 
Интересно е обаче да отбележим, че профила на разпространение на топлината в ,,студеното`` тяло е гладък и има форма наподобяваща $e^{-x}$. Тук топлината ,,навлязла`` в студеното тяло се разпространява бързо и ,,топлото`` тяло не е с постоянна температура навсякъде в областта.
Подобна ситуация наблюдаваме и при грешката - има асиметричен профил, отново е по-голяма при по-студеното тяло, но не е толкова асиметрична, колкото в разгледания граничен случай в началото на секцията.

